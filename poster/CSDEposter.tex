%%%%%%%%%%%%%%%%%%%%%%%%%%%%%%%%%%%%%%%%%
% a0poster Portrait Poster
% LaTeX Template
% Version 1.0 (22/06/13)
%
% The a0poster class was created by:
% Gerlinde Kettl and Matthias Weiser (tex@kettl.de)
%
% This template has been downloaded from:
% http://www.LaTeXTemplates.com
%
% License:
% CC BY-NC-SA 3.0 (http://creativecommons.org/licenses/by-nc-sa/3.0/)
%
%%%%%%%%%%%%%%%%%%%%%%%%%%%%%%%%%%%%%%%%%

%----------------------------------------------------------------------------------------
%	PACKAGES AND OTHER DOCUMENT CONFIGURATIONS
%----------------------------------------------------------------------------------------

\documentclass[a0,portrait]{a0poster}

\usepackage{multicol} % This is so we can have multiple columns of text side-by-side
\columnsep=100pt % This is the amount of white space between the columns in the poster
\columnseprule=3pt % This is the thickness of the black line between the columns in the poster

\usepackage[svgnames]{xcolor} % Specify colors by their 'svgnames', for a full list of all colors available see here: http://www.latextemplates.com/svgnames-colors

\usepackage{times} % Use the times font
%\usepackage{palatino} % Uncomment to use the Palatino font

\usepackage{graphicx} % Required for including images
\graphicspath{{figures/}} % Location of the graphics files
\usepackage{booktabs} % Top and bottom rules for table
\usepackage[font=small,labelfont=bf]{caption} % Required for specifying captions to tables and figures
\usepackage{amsfonts, amsmath, amsthm, amssymb} % For math fonts, symbols and environments
\usepackage{wrapfig} % Allows wrapping text around tables and figures

\begin{document}

%----------------------------------------------------------------------------------------
%	POSTER HEADER
%----------------------------------------------------------------------------------------

% The header is divided into two boxes:
% The first is 67% wide and houses the title, subtitle, names, university/organization and contact information
% The second is 33% wide and houses a logo for your university/organization or a photo of you
% The widths of these boxes can be easily edited to accommodate your content as you see fit

\begin{minipage}[b]{0.67\linewidth}
\veryHuge \color{Crimson} \textbf{[Your CSDE Poster Main Title That Can Be Really Long, or Really Short]} \color{Black}\\ % Title
\Huge\textit{[Subtitle: It's up to you]}\\[2cm] % Subtitle
\huge \textbf{[Author - Department]}\\[0.5cm] % Author(s)
\huge University of Washington\\[0.4cm] % University/organization
\Large email@uw.edu\\
\end{minipage}
%
\begin{minipage}[b]{0.33\linewidth}
\includegraphics[width=25cm]{/home/nyanyan/Documents/MXU5MR/poster/figures/CSDE_logo_transparent.png}\\
\end{minipage}

\vspace{1cm} % A bit of extra whitespace between the header and poster content

%%% N


%----------------------------------------------------------------------------------------

\begin{multicols}{2} % This is how many columns your poster will be broken into, a portrait poster is generally split into 2 columns

%----------------------------------------------------------------------------------------
%	ABSTRACT
%----------------------------------------------------------------------------------------

\color{Navy} % Navy color for the abstract

\begin{abstract}

Mexico has had dramatic declines in under 5 mortality from 1990 to 2015 and is one of only a handful of countries that have reached the Millennium Development Goal 4 (MDG4) of reducing national level mortality by two thirds from 48 to 16 deaths per 1000 live births. While national level reports of mortality have been made by several international groups and have tracked the changes in Mexico's under  mortality, no study to date has done a comparative assessment of the differential levels of child mortality within Mexico by municipality. In order to assess how municipalities differ in their under 5 mortality rate and probability of death before age 5 ($~_{5}q_{0}$) we calculate age specific mortality rate using spatiotemporal age models. In addition, we assess the difference in time from birth till registration by municipality for all births recorded in the years 2000-2015 in order to assess quality of vital registration.

Under 5 mortality count data were taken from registrations reported by Subsistema de Información sobre Nacimientos (SINAC) and Instituto Nacional de Estadística y Geografía (INEGI) while population data was calculated from birth record data from each district as reported by INEGI. Mortality counts were estimated from this data using a poisson hierarchical spatiotemporal age regression where the relative risk was used as the mortality rate estimate for a given municipality.

We found that estimates of under 5 mortality are largely improved by the use of spatiotemporal age models. Despite making large strides in improving national level under 5 mortality, many municipalities have not hit the national rate of less than 16 deaths per 1000 live births, with some municipalities having values averaging above the 1990 reference value for the MDG4 target. We also found large differences by municipality in the time from birth to registration as well as significant spatial patterning in these delays that likely bias estimates such that they show less mortality than there actually is within the country.

Though are analysis also showed Mexico making major declines in national level mortality between 2000 and 2015, are study found a significant number of municipalities who had not reached this goal by 2015. In addition we found significant geographic correlation in birth registry delays that likely bias our findings towards more optimistic results of child mortality and inequities. We found strong evidence that municipalities in Mexico have experienced the improvements to health differentially and in a geographically stratified way.


\end{abstract}

% %----------------------------------------------------------------------------------------
% %	INTRODUCTION
% %----------------------------------------------------------------------------------------
%
% \color{black} % black color for the introduction
%
% \section*{Introduction}
%
% Aliquam non lacus dolor, \textit{a aliquam quam} \cite{Smith:2012qr}. Cum sociis natoque penatibus et magnis dis parturient montes, nascetur ridiculus mus. Nulla in nibh mauris. Donec vel ligula nisi, a lacinia arcu. Sed mi dui, malesuada vel consectetur et, egestas porta nisi. Sed eleifend pharetra dolor, et dapibus est vulputate eu. \textbf{Integer faucibus elementum felis vitae fringilla.} In hac habitasse platea dictumst. Duis tristique rutrum nisl, nec vulputate elit porta ut. Donec sodales sollicitudin turpis sed convallis. Etiam mauris ligula, blandit adipiscing condimentum eu, dapibus pellentesque risus.
%
% \textit{Aliquam auctor}, metus id ultrices porta, risus enim cursus sapien, quis iaculis sapien tortor sed odio. Mauris ante orci, euismod vitae tincidunt eu, porta ut neque. Aenean sapien est, viverra vel lacinia nec, venenatis eu nulla. Maecenas ut nunc nibh, et tempus libero. Aenean vitae risus ante. Pellentesque condimentum dui. Etiam sagittis purus non tellus tempor volutpat. Donec et dui non massa tristique adipiscing.

%----------------------------------------------------------------------------------------
%	OBJECTIVES
%----------------------------------------------------------------------------------------

\color{black} % DarkSlateGray color for the rest of the content

\section*{Main Objectives}

\begin{enumerate}
\item Understand the difference in quality of Vital Registration Data at the municipality level.
\item Test the geographic and temporal correlation of child mortality data.
\item Use 2D Matern covariance specification for geographic correlations in a novel manner, and test against more traditional methods.
\item Build a fulll time series of estimates for age specific child mortality and$~_{5}q_{0}$ for all 2456 municipalities for years 2000-2015.
\item Assess geographic and tempor patterns in results.
\end{enumerate}

%----------------------------------------------------------------------------------------
%	MATERIALS AND METHODS
%----------------------------------------------------------------------------------------

\section*{Materials and Methods}

De-identified person level data was extracted from INEGI and SINAC vital registration reports for the years 2000 to 2015 for birth records. Of the 43594696 birth records that were extracted 99.65\% had a properly identified household location of parents that was used as the geolocation for this analysis, placing each record in one of the 2456 municipalities within Mexico. The records that did not have a municipality associated with them were approximately evenly distributed across states and were discarded from the analysis. Death registration data was extracted from person level de-identified INEGI reports where municipality of residence, year of occurrence, and age of individual given by single year was used provided that information existed for the death record. Of the 592274 deaths that were reported as under the age of 5, 99.72\% had values for year, age, and municipality that were valid. Since data was not linked by person and single year populations by single year age and municipality were not available in other data sources, populations were estimated by taking the births of individuals for a particular municipality age and moving them one year forward in age as calendar year progressed, while also subtracting the deaths for that population. The formula is as follows where $P$ is population number, $D$ is death number, $a$ is a single year age, $t$ is a single year time period, and $l$ is a municipality.

$$
P_{l,a,t} = P_{l,a-1,t-1} - D_{l,a-1,t-1}
$$

This cohort method of tracking mortality and populations makes the assumption of net zero migration for all demographic units.
%------------------------------------------------

\subsection*{Mathematical Section}

Birth records were analyzed for spatial correlation of quality (time till birth recorded) by use of a Moran's I test.

$$
I = \frac{N}{W} ~ \frac{\Sigma_i \Sigma_j w_{ij}(x_i - \bar{x})(x_j - \bar{x})}{\Sigma_i (x_i - \bar{x})^2}
$$

Age specific mortality was estimated using the following poisson mixed effect model structure.

\begin{flalign*}
  D_{l,a,t} & \sim Poisson(\hat{D}_{l,a,t}) \\
  \hat{D}_{l,a,t} & = exp(\beta_a + \phi_{l,a,t}) * P_{l,a,t} \\
  \phi & \sim \text{MVN}(\mathbf{0}, Q^{-1}) \\
  Q & = Q^t \otimes Q^a \otimes Q^l \\
\end{flalign*}

Where $Q^t$ and $Q^a$ follow a multivariate normal distribution with precision.

\begin{aligned}
Q^{AR}_{i,j} =
\begin{cases}
    \frac{1}{\sigma^2} ,& \text{if  } i = j = 0 | i = j = max(i) \\
    \frac{1 + \rho^2}{\sigma^2} ,& \text{else if  } i = j \\
    \frac{-\rho}{\sigma^2},  & \text{else if  } i \sim j \\
    0, & \text{otherwise}
\end{cases}
 \end{aligned}

and $Q^l$ follows either a LCAR or 2D Matern specification. This will give us our matrix $Q$ which forms the structure of our approximately 200,000 random effects (one for each unique combination of municipality(2456), year(16), and age group(5)).
%----------------------------------------------------------------------------------------
%	RESULTS
%----------------------------------------------------------------------------------------

\section*{Results}

Donec faucibus purus at tortor egestas eu fermentum dolor facilisis. Maecenas tempor dui eu neque fringilla rutrum. Mauris \emph{lobortis} nisl accumsan. Aenean vitae risus ante.
%
\begin{wraptable}{l}{12cm} % Left or right alignment is specified in the first bracket, the width of the table is in the second
\begin{tabular}{l l l}
\toprule
\textbf{Treatments} & \textbf{Response 1} & \textbf{Response 2}\\
\midrule
Treatment 1 & 0.0003262 & 0.562 \\
Treatment 2 & 0.0015681 & 0.910 \\
Treatment 3 & 0.0009271 & 0.296 \\
\bottomrule
\end{tabular}
\captionof{table}{\color{Green} Table caption}
\end{wraptable}
%
Phasellus imperdiet, tortor vitae congue bibendum, felis enim sagittis lorem, et volutpat ante orci sagittis mi. Morbi rutrum laoreet semper. Morbi accumsan enim nec tortor consectetur non commodo nisi sollicitudin. Proin sollicitudin. Pellentesque eget orci eros. Fusce ultricies, tellus et pellentesque fringilla, ante massa luctus libero, quis tristique purus urna nec nibh.

Nulla ut porttitor enim. Suspendisse venenatis dui eget eros gravida tempor. Mauris feugiat elit et augue placerat ultrices. Morbi accumsan enim nec tortor consectetur non commodo. Pellentesque condimentum dui. Etiam sagittis purus non tellus tempor volutpat. Donec et dui non massa tristique adipiscing. Quisque vestibulum eros eu. Phasellus imperdiet, tortor vitae congue bibendum, felis enim sagittis lorem, et volutpat ante orci sagittis mi. Morbi rutrum laoreet semper. Morbi accumsan enim nec tortor consectetur non commodo nisi sollicitudin.

\begin{center}\vspace{1cm}
\includegraphics[width=0.4\linewidth]{/home/nyanyan/Documents/MXU5MR/analysis/plots/regisdiff.png}
\captionof{figure}{\color{Green} Figure caption}
\end{center}\vspace{1cm}

In hac habitasse platea dictumst. Etiam placerat, risus ac.

Adipiscing lectus in magna blandit:

\begin{center}\vspace{1cm}
\begin{tabular}{l l l l}
\toprule
\textbf{Treatments} & \textbf{Response 1} & \textbf{Response 2} \\
\midrule
Treatment 1 & 0.0003262 & 0.562 \\
Treatment 2 & 0.0015681 & 0.910 \\
Treatment 3 & 0.0009271 & 0.296 \\
\bottomrule
\end{tabular}
\captionof{table}{\color{Green} Table caption}
\end{center}\vspace{1cm}

Vivamus sed nibh ac metus tristique tristique a vitae ante. Sed lobortis mi ut arcu fringilla et adipiscing ligula rutrum. Aenean turpis velit, placerat eget tincidunt nec, ornare in nisl. In placerat.

\begin{center}\vspace{1cm}
\includegraphics[width=0.8\linewidth]{placeholder}
\captionof{figure}{\color{Green} Figure caption}
\end{center}\vspace{1cm}

%----------------------------------------------------------------------------------------
%	CONCLUSIONS
%----------------------------------------------------------------------------------------

\color{black} % black color for the conclusions to make them stand out

\section*{Conclusions}

\begin{itemize}
\item Pellentesque eget orci eros. Fusce ultricies, tellus et pellentesque fringilla, ante massa luctus libero, quis tristique purus urna nec nibh. Phasellus fermentum rutrum elementum. Nam quis justo lectus.
\item Vestibulum sem ante, hendrerit a gravida ac, blandit quis magna.
\item Donec sem metus, facilisis at condimentum eget, vehicula ut massa. Morbi consequat, diam sed convallis tincidunt, arcu nunc.
\item Nunc at convallis urna. isus ante. Pellentesque condimentum dui. Etiam sagittis purus non tellus tempor volutpat. Donec et dui non massa tristique adipiscing.
\end{itemize}

\color{DarkSlateGray} % Set the color back to DarkSlateGray for the rest of the content

%----------------------------------------------------------------------------------------
%	FORTHCOMING RESEARCH
%----------------------------------------------------------------------------------------

\section*{Forthcoming Research}

Vivamus molestie, risus tempor vehicula mattis, libero arcu volutpat purus, sed blandit sem nibh eget turpis. Maecenas rutrum dui blandit lorem vulputate gravida. Praesent venenatis mi vel lorem tempor at varius diam sagittis. Nam eu leo id turpis interdum luctus a sed augue. Nam tellus.

 %----------------------------------------------------------------------------------------
%	REFERENCES
%----------------------------------------------------------------------------------------

%\nocite{*} % Print all references regardless of whether they were cited in the poster or not
\bibliographystyle{plain} % Plain referencing style
\bibliography{sample} % Use the example bibliography file sample.bib

%----------------------------------------------------------------------------------------
%	ACKNOWLEDGEMENTS
%----------------------------------------------------------------------------------------

\section*{Acknowledgements}

Etiam fermentum, arcu ut gravida fringilla, dolor arcu laoreet justo, ut imperdiet urna arcu a arcu. Donec nec ante a dui tempus consectetur. Cras nisi turpis, dapibus sit amet mattis sed, laoreet.

%----------------------------------------------------------------------------------------

\end{multicols}
\end{document}
