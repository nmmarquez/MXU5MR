%%%%%%%%%%%%%%%%%%%%%%%%%%%%%%%%%%%%%%%%%
% a0poster Portrait Poster
% LaTeX Template
% Version 1.0 (22/06/13)
%
% The a0poster class was created by:
% Gerlinde Kettl and Matthias Weiser (tex@kettl.de)
%
% This template has been downloaded from:
% http://www.LaTeXTemplates.com
%
% License:
% CC BY-NC-SA 3.0 (http://creativecommons.org/licenses/by-nc-sa/3.0/)
%
%%%%%%%%%%%%%%%%%%%%%%%%%%%%%%%%%%%%%%%%%

%----------------------------------------------------------------------------------------
%	PACKAGES AND OTHER DOCUMENT CONFIGURATIONS
%----------------------------------------------------------------------------------------

\documentclass[a0,portrait]{a0poster}

\usepackage{multicol} % This is so we can have multiple columns of text side-by-side
\columnsep=100pt % This is the amount of white space between the columns in the poster
\columnseprule=3pt % This is the thickness of the black line between the columns in the poster

\usepackage[svgnames]{xcolor} % Specify colors by their 'svgnames', for a full list of all colors available see here: http://www.latextemplates.com/svgnames-colors

\usepackage{times} % Use the times font
%\usepackage{palatino} % Uncomment to use the Palatino font

\usepackage{graphicx} % Required for including images
\graphicspath{{figures/}} % Location of the graphics files
\usepackage{booktabs} % Top and bottom rules for table
\usepackage[font=small,labelfont=bf]{caption} % Required for specifying captions to tables and figures
\usepackage{amsfonts, amsmath, amsthm, amssymb} % For math fonts, symbols and environments
\usepackage{wrapfig} % Allows wrapping text around tables and figures

\begin{document}

%----------------------------------------------------------------------------------------
%	POSTER HEADER
%----------------------------------------------------------------------------------------

% The header is divided into two boxes:
% The first is 67% wide and houses the title, subtitle, names, university/organization and contact information
% The second is 33% wide and houses a logo for your university/organization or a photo of you
% The widths of these boxes can be easily edited to accommodate your content as you see fit

\begin{minipage}[b]{0.67\linewidth}
\veryHuge \color{Crimson} \textbf{Under 5 Mortality Estimates For Mexico's ~~~~~~Municipalities: 2000-2015} \color{Black}\\ % Title
\Huge\textit{Estimating Small Area Changes in Child Mortality over Time in Mexico}\\[2cm] % Subtitle
\huge \textbf{Neal Marquez - Sociology}\\[0.5cm] % Author(s)
\huge University of Washington\\[0.4cm] % University/organization
\Large nmarquez@uw.edu\\
\end{minipage}
%
\begin{minipage}[b]{0.33\linewidth}
\includegraphics[width=25cm]{/home/nyannyan/Documents/MXU5MR/poster/figures/CSDE_logo_transparent.png}\\
\end{minipage}

\vspace{1cm} % A bit of extra whitespace between the header and poster content

%%% N


%----------------------------------------------------------------------------------------

\begin{multicols}{2} % This is how many columns your poster will be broken into, a portrait poster is generally split into 2 columns

%----------------------------------------------------------------------------------------
%	ABSTRACT
%----------------------------------------------------------------------------------------

\color{Navy} % Navy color for the abstract

\begin{abstract}

Mexico has had dramatic declines in under 5 mortality from 1990 to 2015 and is one of only a handful of countries that have reached the Millennium Development Goal 4 (MDG4) of reducing national level mortality by two thirds from 48 to 16 deaths per 1000 live births. While national level reports of mortality have been made by several international groups and have tracked the changes in Mexico's under  mortality, no study to date has done a comparative assessment of the differential levels of child mortality within Mexico by municipality. In order to assess how municipalities differ in their under 5 mortality rate and probability of death before age 5 ($~_{5}q_{0}$) we calculate age specific mortality rate using spatiotemporal age models. In addition, we assess the difference in time from birth till registration by municipality for all births recorded in the years 2000-2015 in order to assess quality of vital registration.

\end{abstract}

% %----------------------------------------------------------------------------------------
% %	INTRODUCTION
% %----------------------------------------------------------------------------------------
%
% \color{black} % black color for the introduction
%
% \section*{Introduction}
%
% Aliquam non lacus dolor, \textit{a aliquam quam} \cite{Smith:2012qr}. Cum sociis natoque penatibus et magnis dis parturient montes, nascetur ridiculus mus. Nulla in nibh mauris. Donec vel ligula nisi, a lacinia arcu. Sed mi dui, malesuada vel consectetur et, egestas porta nisi. Sed eleifend pharetra dolor, et dapibus est vulputate eu. \textbf{Integer faucibus elementum felis vitae fringilla.} In hac habitasse platea dictumst. Duis tristique rutrum nisl, nec vulputate elit porta ut. Donec sodales sollicitudin turpis sed convallis. Etiam mauris ligula, blandit adipiscing condimentum eu, dapibus pellentesque risus.
%
% \textit{Aliquam auctor}, metus id ultrices porta, risus enim cursus sapien, quis iaculis sapien tortor sed odio. Mauris ante orci, euismod vitae tincidunt eu, porta ut neque. Aenean sapien est, viverra vel lacinia nec, venenatis eu nulla. Maecenas ut nunc nibh, et tempus libero. Aenean vitae risus ante. Pellentesque condimentum dui. Etiam sagittis purus non tellus tempor volutpat. Donec et dui non massa tristique adipiscing.

%----------------------------------------------------------------------------------------
%	OBJECTIVES
%----------------------------------------------------------------------------------------

\color{black} % DarkSlateGray color for the rest of the content

\section*{Main Objectives}

\begin{enumerate}
\item Understand the difference in quality of Vital Registration Data at the municipality level.
\item Build a fulll time series of estimates for age specific child mortality and$~_{5}q_{0}$ for all 2456 municipalities for years 2000-2015.
\item Assess changes in distribution of$~_{5}q_{0}$ experience.
\item Decompose the relative contribution made from real change and compositioinal changes of$~_{5}q_{0}$.
\end{enumerate}

%----------------------------------------------------------------------------------------
%	MATERIALS AND METHODS
%----------------------------------------------------------------------------------------

\section*{Materials and Methods}

De-identified person level data was extracted from INEGI and SINAC vital registration reports for the years 2000 to 2015 for birth records. Of the 43594696 birth records that were extracted 99.65\% had a properly identified household location of parents that was used as the geolocation for this analysis, placing each record in one of the 2456 municipalities within Mexico. The records that did not have a municipality associated with them were approximately evenly distributed across states and were discarded from the analysis. Death registration data was extracted from person level de-identified INEGI reports where municipality of residence, year of occurrence, and age of individual given by single year was used provided that information existed for the death record. Of the 592274 deaths that were reported as under the age of 5, 99.72\% had values for year, age, and municipality that were valid. Since data was not linked by person and single year populations by single year age and municipality were not available in other data sources, populations were estimated by taking the births of individuals for a particular municipality age and moving them one year forward in age as calendar year progressed, while also subtracting the deaths for that population. The formula is as follows where $P$ is population number, $D$ is death number, $a$ is a single year age, $t$ is a single year time period, and $l$ is a municipality.

$$
P_{l,a,t} = P_{l,a-1,t-1} - D_{l,a-1,t-1}
$$

%This cohort method of tracking mortality and populations makes the assumption of net zero migration for all demographic units.
%------------------------------------------------

%----------------------------------------------------------------------------------------
%	RESULTS
%----------------------------------------------------------------------------------------

\section*{Results}

Analysis of time from birth until registration showed that there was a significant amount of variability in the distribution of births at each municipality with a mean time of .6  and a standard deviation of .4. In addition to this variability there was a strong geographic correlation between the time delay of births until time to registration with a Moran's I of .7661 (p $<$ .01) indicative of strong positive geographic autocorrelation(figure 1).

Results from a multidimensional,age, time and geography, Gaussian Markov Random Field latent model were used in the final estimation of municipality age specific child mortality and used to estimate$~_{5}q_{0}$ at the municipality level. Anaysis of$~_{5}q_{0}$ found that there was a significant decline in the level of inequality of lived$~_{5}q_{0}$ though all metrics showed this level stabalizing in recent years. Though inequality dcereased in the observed time series the relative positioning of municipalities were strongly correlated between 2000 and 2015 , 0.6232 (0.5903-0.6549), suggesting that municipalities that had relatively poor child mortality as measured by$~_{5}q_{0}$ within Mexico in 2000 continued to perform poorly relative to the national average in 2015.

Decomposing changes to the national level$~_{5}q_{0}$ into compositional changes and real rate changes found that 99.5\% of the changes to national$~_{5}q_{0}$ came from real declines in rates among Mexican municipalities. Around 50\% of the decline observed between 2000 and 2015 was made by only 20\% of the population, which likely contributed to the decline in measures of inequality. Furthermore, ordering the municipalities from the largest to the smallest organizational units, we found that a nearly equal contribution to the change in national level$~_{5}q_{0}$ came from the smallest 50\% and the largest 50\% of the municipalities. The municipalities that most hindered the decline in national$~_{5}q_{0}$ were largely due to increases in the population size and their relative contribution to the national average.


%
\begin{center}\vspace{1cm}
\includegraphics[width=.8\columnwidth]{/home/nyannyan/Documents/MXU5MR/analysis/plots/regDelayMap.png}
\captionof{figure}{\color{Green} Map of Delay in Registration Time}
\end{center}\vspace{1cm}
%

%
\begin{center}\vspace{1cm}
\includegraphics[width=.7\columnwidth]{/home/nyannyan/Documents/MXU5MR/analysis/plots/ineq.png}
\captionof{figure}{\color{Green} Changes in Inequality & Contribution}
\end{center}\vspace{1cm}
%

%
\begin{center}\vspace{1cm}
\includegraphics[width=.8\columnwidth]{/home/nyannyan/Documents/MXU5MR/analysis/plots/lisaplot.png}
\captionof{figure}{\color{Green} Clusters of Significant Change and Level}
\end{center}\vspace{1cm}
%

%
\begin{center}\vspace{1cm}
\includegraphics[width=.8\columnwidth]{/home/nyannyan/Documents/MXU5MR/analysis/plots/Oaxaca.png}
\captionof{figure}{\color{Green} Oaxaca in Detail}
\end{center}\vspace{1cm}
%

Examing the lowest and highest 30 municipalities of$~_{5}q_{0}$ found that Oaxaca, which contains around 25\% of the countries municipalities, makes up 43\% of the highest observed$~_{5}q_{0}$ municipalities as well as 53\% of the lowest observed$~_{5}q_{0}$ municipalities. Looking at just Oaxaca, the state level inequalities have not significantly changed over the observed time period. Other low performers in terms of relative level of 2015$~_{5}q_{0}$ come from the states of Chiapas, Guerrero, and Chihuahua while high performing municipalities are located in Tamaulipas, Yucatan, and Michoacan de Ocampo.
%----------------------------------------------------------------------------------------
%	CONCLUSIONS
%----------------------------------------------------------------------------------------

\color{black} % black color for the conclusions to make them stand out

\section*{Conclusions}

\begin{itemize}
\item Birth vital registration quality is heterogenous across Mexico and is significantly spatially correlated.
\item Additional evidence of reducing child mortality and inequality at national level.
\item Reduction has largely come from real changes in rate rather than compositional changes.
\item Mexico has stagnated in reducing inequality of child mortality in both relative and absolute measures.
\item State level performance of reduction and inequality of$~_{5}q_{0}$ measures is heterogenous.
\end{itemize}

\color{DarkSlateGray} % Set the color back to DarkSlateGray for the rest of the content

%----------------------------------------------------------------------------------------
%	FORTHCOMING RESEARCH
%----------------------------------------------------------------------------------------

% \section*{Forthcoming Research}
%
% Future research will address the differences in quality of vital registration with increased variance of estimates as well as estimate the effect that migration has on mortality parameter estimates.

%----------------------------------------------------------------------------------------
%	REFERENCES
%----------------------------------------------------------------------------------------

% \section*{References}
% \begin{itemize}
% \item Wang H, Naghavi M, Allen C et al. Global, regional, and national life expectancy, all-cause mortality, and cause-specific mortality for 249 causes of death, 1980-2015: A systematic analysis for the global burden of disease study 2015. The Lancet 2016; 388: 1459–544
% \item Victora CG, Requejo JH, Barros AJD et al. Countdown to 2015: A decade of tracking progress for maternal, newborn, and child survival. The Lancet 2016; 387: 2049-59.
% \item Dwyer-Lindgren L, Kakungu F, Hangoma P et al. Estimation of district-level under-5 mortality in zambia using birth history data, 1980-2010. Spatial and Spatio-temporal Epidemiology 2014; 11: 89-107
% \end{itemize}

%----------------------------------------------------------------------------------------
%	ACKNOWLEDGEMENTS
%----------------------------------------------------------------------------------------

\section*{Acknowledgements}

This research would have not been possible without the advice and guidence of the people at Instituto Nacional de Salud Publica, the Institute for Health Metrics and Evaluation, and the Department of Global Health at the University of Washington.

%----------------------------------------------------------------------------------------

\end{multicols}
\end{document}
